%\RequirePackage[l2tabu]{nag}		% Warns for incorrect (obsolete) LaTeX usage
%
%
%
% Memoir is a flexible class for typesetting poetry, fiction,
% non-fiction and mathematical works as books, reports, articles or
% manuscripts. CTAN repository is found at:
% http://www.ctan.org/tex-archive/macros/latex/contrib/memoir/
%
%
% UoB guidelines for thesis presentation were found at:
% http://www.bris.ac.uk/esu/pg/pgrcop11-12topic.pdf#page=49
%
% UoB guidelines:
%
% The dissertation must be printed on A4 white paper. Paper up to A3 may be used
% for maps, plans, diagrams and illustrative material. Pages (apart from the
% preliminary pages) should normally be double-sided.
%
% Memoir class loads useful packages by default (see manual).
\documentclass[a4paper,12pt,leqno,openbib,oldfontcommands,spanish]{memoir} %add 'draft' to turn draft option on (see below)
%
%
% Adding metadata:
\DisemulatePackage{setspace}
\usepackage{setspace}
\usepackage[bottom,multiple]{footmisc}
\usepackage{wrapfig}
\usepackage{fancybox}
\usepackage{multicol}
\usepackage{caption}
\usepackage{datetime}
%\usepackage{fancyref}
\usepackage{ifpdf}
\ifpdf
\pdfinfo{
   /Author (Fulanito)
   /Title (PhD Thesis)
   /Keywords (One; Two;Three)
   /CreationDate (D:\pdfdate)
}
\fi
% When draft option is on.
\ifdraftdoc
	\usepackage{draftwatermark}				%Sets watermarks up.
	\SetWatermarkScale{0.3}
	\SetWatermarkText{\bf Draft: \today}
\fi

\usepackage[pass]{geometry}

%
% Declare figure/table as a subfloat.
\newsubfloat{figure}
\newsubfloat{table}
% Better page layout for A4 paper, see memoir manual.
\settrimmedsize{297mm}{210mm}{*}
\setlength{\trimtop}{0pt}
\setlength{\trimedge}{\stockwidth}
\addtolength{\trimedge}{-\paperwidth}
\settypeblocksize{634pt}{448.13pt}{*}
\setulmargins{4cm}{*}{*}
%\setlrmargins{*}{*}{1.5} %original
\setlrmargins{*}{*}{0.70}
\setmarginnotes{17pt}{51pt}{\onelineskip}
\setheadfoot{\onelineskip}{2\onelineskip}
\setheaderspaces{*}{2\onelineskip}{*}
\checkandfixthelayout
%
\frenchspacing
% Font with math support: New Century Schoolbook
\usepackage{fouriernc}
\usepackage[T1]{fontenc}

%
% UoB guidelines:
%
% Text should be in double or 1.5 line spacing, and font size should be
% chosen to ensure clarity and legibility for the main text and for any
% quotations and footnotes. Margins should allow for eventual hard binding.
%
% Note: This is automatically set by memoir class. Nevertheless \OnehalfSpacing
% enables double spacing but leaves single spaced for captions for instance.
\OnehalfSpacing
%
% Sets numbering division level
\setsecnumdepth{subsection}
\maxsecnumdepth{subsubsection}
%
% Chapter style (taken and slightly modified from Lars Madsen Memoir Chapter
% Styles document
\usepackage{calc,soul,fourier}
\makeatletter
\newlength\dlf@normtxtw
\setlength\dlf@normtxtw{\textwidth}
\newsavebox{\feline@chapter}
\newcommand\feline@chapter@marker[1][4cm]{%
	\sbox\feline@chapter{%
		\resizebox{!}{#1}{\fboxsep=1pt%
			\colorbox{gray}{\color{white}\thechapter}%
		}}%
		\rotatebox{90}{%
			\resizebox{%
				\heightof{\usebox{\feline@chapter}}+\depthof{\usebox{\feline@chapter}}}%
			{!}{\scshape\so\@chapapp}}\quad%
		\raisebox{\depthof{\usebox{\feline@chapter}}}{\usebox{\feline@chapter}}%
}
\newcommand\feline@chm[1][4cm]{%
	\sbox\feline@chapter{\feline@chapter@marker[#1]}%
	\makebox[0pt][c]{% aka \rlap
		\makebox[1cm][r]{\usebox\feline@chapter}%
	}}
\makechapterstyle{daleifmodif}{
	\renewcommand\chapnamefont{\normalfont\Large\scshape\raggedleft\so}
	\renewcommand\chaptitlefont{\normalfont\Large\bfseries\scshape}
	\renewcommand\chapternamenum{} \renewcommand\printchaptername{}
	\renewcommand\printchapternum{\null\hfill\feline@chm[3.5cm]\par}
	\renewcommand\afterchapternum{\par\vskip\midchapskip}
	\renewcommand\printchaptertitle[1]{\color{gray}\chaptitlefont\raggedleft ##1\par}
}
\makeatother
\chapterstyle{daleifmodif}
%
% UoB guidelines:
%
% The pages should be numbered consecutively at the bottom centre of the
% page.
\makepagestyle{myvf}
\makeoddfoot{myvf}{}{\thepage}{}
\makeevenfoot{myvf}{}{\thepage}{}
\makeheadrule{myvf}{\textwidth}{\normalrulethickness}
\makeevenhead{myvf}{\small\textsc{\leftmark}}{}{}
\makeoddhead{myvf}{}{}{\small\textsc{\rightmark}}
\pagestyle{myvf}
%
% Oscar's command (it works):
% Fills blank pages until next odd-numbered page. Used to emulate single-sided
% frontmatter. This will work for title, abstract and declaration. Though the
% contents sections will each start on an odd-numbered page they will
% spill over onto the even-numbered pages if extending beyond one page
% (hopefully, this is ok).
\newcommand{\clearemptydoublepage}{\newpage{\thispagestyle{empty}\cleardoublepage}}
%
%
% Creates indexes for Table of Contents, List of Figures, List of Tables and Index
\makeindex
% \printglossaries below creates a list of abbreviations. \gls and related
% commands are then used throughout the text, so that latex can automatically
% keep track of which abbreviations have already been defined in the text.
%
% The import command enables each chapter tex file to use relative paths when
% accessing supplementary files. For example, to include
% chapters/brewing/images/figure1.png from chapters/brewing/brewing.tex we can
% use
% \includegraphics{images/figure1}
% instead of
% \includegraphics{chapters/brewing/images/figure1}
\usepackage{import}
% Add other packages needed for chapters here. For example:
\usepackage{lipsum}					%Needed to create dummy text
\usepackage{amsfonts} 					%Calls Amer. Math. Soc. (AMS) fonts
\usepackage[centertags]{amsmath}			%Writes maths centred down
\usepackage{stmaryrd}					%New AMS symbols
\usepackage{amssymb}					%Calls AMS symbols
\usepackage{amsthm}					%Calls AMS theorem environment
\usepackage{newlfont}					%Helpful package for fonts and symbols
\usepackage{layouts}					%Layout diagrams
\usepackage{graphicx}					%Calls figure environment
\usepackage{longtable,rotating}			%Long tab environments including rotation.
\usepackage[utf8]{inputenc}			%Needed to encode non-english characters
									%directly for mac
\usepackage{colortbl}					%Makes coloured tables
\usepackage{wasysym}					%More math symbols
\usepackage{mathrsfs}					%Even more math symbols
\usepackage{float}						%Helps to place figures, tables, etc.
\usepackage{verbatim}					%Permits pre-formated text insertion
\usepackage{upgreek }					%Calls other kind of greek alphabet
\usepackage{latexsym}					%Extra symbols
\usepackage[square,numbers,sort&compress]{natbib}%Calls bibliography commands
\usepackage[hyphens]{url}						%Supports url commands
% \usepackage{etex}						%eTeXÕs extended support for counters
% \usepackage{fixltx2e}					%Eliminates some in felicities of the
									%original LaTeX kernel
\usepackage[spanish,es-tabla]{babel}		%For languages characters and hyphenation
\addto{\captionsenglish}{%
  \renewcommand{\bibname}{Bibliografía}
}
\usepackage{color}                    				%Creates coloured text and background
\usepackage[colorlinks=true, allcolors=black]{hyperref}              %Creates hyperlinks in cross references
\usepackage{memhfixc}					%Must be used on memoir document
									%class after hyperref
\usepackage{enumerate}					%For enumeration counter
\usepackage{footnote}					%For footnotes
\usepackage{microtype}					%Makes pdf look better.
\usepackage{rotfloat}					%For rotating and float environments as tables,
									%figures, etc.
\usepackage{alltt}						%LaTeX commands are not disabled in
									%verbatim-like environment
\usepackage[version=0.96]{pgf}			%PGF/TikZ is a tandem of languages for producing vector graphics from a
\usepackage{tikz}						%geometric/algebraic description.
\usetikzlibrary{arrows,shapes,snakes,
		       automata,backgrounds,
		       petri,topaths}				%To use diverse features from tikz
%
%Reduce widows  (the last line of a paragraph at the start of a page) and orphans
% (the first line of paragraph at the end of a page)
\widowpenalty=1000
\clubpenalty=1000
%
% New command definitions for my thesis
%
\newcommand{\keywords}[1]{\par\noindent{\small{\bf Keywords:} #1}} %Defines keywords small section
\newcommand{\parcial}[2]{\frac{\partial#1}{\partial#2}}                             %Defines a partial operator
\newcommand{\vectorr}[1]{\mathbf{#1}}                                                        %Defines a bold vector
\newcommand{\vecol}[2]{\left(                                                                         %Defines a column vector
	\begin{array}{c}
		\displaystyle#1 \\
		\displaystyle#2
	\end{array}\right)}
\newcommand{\mados}[4]{\left(                                                                       %Defines a 2x2 matrix
	\begin{array}{cc}
		\displaystyle#1 &\displaystyle #2 \\
		\displaystyle#3 & \displaystyle#4
	\end{array}\right)}
\newcommand{\pgftextcircled}[1]{                                                                    %Defines encircled text
    \setbox0=\hbox{#1}%
    \dimen0\wd0%
    \divide\dimen0 by 2%
    \begin{tikzpicture}[baseline=(a.base)]%
        \useasboundingbox (-\the\dimen0,0pt) rectangle (\the\dimen0,1pt);
        \node[circle,draw,outer sep=0pt,inner sep=0.1ex] (a) {#1};
    \end{tikzpicture}
}
\newcommand{\range}[1]{\textnormal{range }#1}                                             %Defines range operator
\newcommand{\innerp}[2]{\left\langle#1,#2\right\rangle}                                 %Defines inner product
\newcommand{\prom}[1]{\left\langle#1\right\rangle}                                         %Defines average operator
\newcommand{\tra}[1]{\textnormal{tra} \: #1}                                                       %Defines trace operator
\newcommand{\sign}[1]{\textnormal{sign\,}#1}                                                   %Defines sign operator
\newcommand{\sech}[1]{\textnormal{sech} #1}                                                  %Defines sech
\newcommand{\diag}[1]{\textnormal{diag} #1}                                                    %Defines diag operator
\newcommand{\arcsech}[1]{\textnormal{arcsech} #1}                                       %Defines arcsech
\newcommand{\arctanh}[1]{\textnormal{arctanh} #1}                                         %Defines arctanh
%Change tombstone symbol
\newcommand{\blackged}{\hfill$\blacksquare$}
\newcommand{\whiteged}{\hfill$\square$}
\newcounter{proofcount}
\renewenvironment{proof}[1][\proofname.]{\par
 \ifnum \theproofcount>0 \pushQED{\whiteged} \else \pushQED{\blackged} \fi%
 \refstepcounter{proofcount}
 \normalfont
 \trivlist
 \item[\hskip\labelsep
       \itshape
   {\bf\em #1}]\ignorespaces
}{%
 \addtocounter{proofcount}{-1}
 \popQED\endtrivlist
}
%
%
% New definition of square root:
% it renames \sqrt as \oldsqrt
\let\oldsqrt\sqrt
% it defines the new \sqrt in terms of the old one
\def\sqrt{\mathpalette\DHLhksqrt}
\def\DHLhksqrt#1#2{%
\setbox0=\hbox{$#1\oldsqrt{#2\,}$}\dimen0=\ht0
\advance\dimen0-0.2\ht0
\setbox2=\hbox{\vrule height\ht0 depth -\dimen0}%
{\box0\lower0.4pt\box2}}
%
% My caption style
\newcommand{\mycaption}[2][\@empty]{
	\captionnamefont{\scshape}
	\changecaptionwidth
	\captionwidth{0.9\linewidth}
	\captiondelim{.\:}
	\indentcaption{0.75cm}
	\captionstyle[\centering]{}
	\setlength{\belowcaptionskip}{10pt}
	\ifx \@empty#1 \caption{#2}\else \caption[#1]{#2}
}
%
% My subcaption style
\newcommand{\mysubcaption}[2][\@empty]{
	\subcaptionsize{\small}
	\hangsubcaption
	\subcaptionlabelfont{\rmfamily}
	\sidecapstyle{\raggedright}
	\setlength{\belowcaptionskip}{10pt}
	\ifx \@empty#1 \subcaption{#2}\else \subcaption[#1]{#2}
}
%
%An initial of the very first character of the content
\usepackage{lettrine}
\newcommand{\initial}[1]{%
	\lettrine[lines=4,lhang=0.33,nindent=0em]{
		\color{gray}
     		{\textsc{#1}}}{}}
%
% Theorem styles used in my thesis
%
\theoremstyle{plain}
\newtheorem{theo}{Theorem}[chapter]
\theoremstyle{plain}
\newtheorem{prop}{Proposition}[chapter]
\theoremstyle{plain}
\theoremstyle{definition}
\newtheorem{dfn}{Definition}[chapter]
\theoremstyle{plain}
\newtheorem{lema}{Lemma}[chapter]
\theoremstyle{plain}
\newtheorem{cor}{Corollary}[chapter]
\theoremstyle{plain}
\newtheorem{resu}{Result}[chapter]
%
% Hyphenation for some words
%
%\hyphenation{e-mail}

\hyphenation{HTTP}
\hyphenation{Internet}
\hyphenation{Johannes}
\hyphenation{Gutenberg}
\hyphenation{res-pec-tively}
\hyphenation{mono-ti-ca-lly}
\hyphenation{hypo-the-sis}
\hyphenation{para-me-ters}
\hyphenation{sol-va-bi-li-ty}

\usepackage{aliascnt}
\newaliascnt{eqfloat}{equation}
\newfloat{eqfloat}{h}{eqflts}
\floatname{eqfloat}{Ecuación}

\newcommand*{\ORGeqfloat}{}
\let\ORGeqfloat\eqfloat
\def\eqfloat{%
  \let\ORIGINALcaption\caption
  \def\caption{%
    \addtocounter{equation}{-1}%
    \ORIGINALcaption
  }%
  \ORGeqfloat
}

%

\newcommand\chapimage[2]{%
\cleartoverso
\thispagestyle{empty}
\noindent
\begin{vplace}[0.6]
%\begin{center}
\begin{figure}[h!]
%\begin{minipage}{\textwidth}
\hspace*{0.11\textwidth}\includegraphics[width=0.75\textwidth]{illustrations/#1}
\captionsetup{justification=centering}
\caption*{#2}
%\end{minipage}
\end{figure}
%\end{center}
\end{vplace}
\clearpage}


%RPM
\newcommand*\widefbox[1]{\fbox{\hspace{1em}#1\hspace{2em}}}
\newcommand{\rpm}{\raisebox{.2ex}{$\scriptstyle\pm$}}

%special cell
\newcommand{\specialcell}[2][c]{%
  \begin{tabular}[#1]{@{}c@{}}#2\end{tabular}}



%table
%\usepackage{dblfnote}
\usepackage{threeparttable}
\usepackage{multirow}
\usepackage{tablefootnote}
\usepackage{makecell}


\usepackage{ragged2e}

%CODE 
\usepackage{listings}
\definecolor{dkgreen}{rgb}{0,0.6,0}
\definecolor{gray}{rgb}{0.5,0.5,0.5}
\definecolor{mauve}{rgb}{0.58,0,0.82}

\colorlet{punct}{red!60!black}
\definecolor{background}{HTML}{FDF7E3}
\definecolor{delim}{RGB}{20,105,176}
\colorlet{numb}{magenta!60!black}

\renewcommand{\lstlistingname}{Nota de código}% Listing -> Algorithm
\renewcommand{\lstlistlistingname}{Notas de código}% List of Listings -> List of Algorithms

\usepackage{array}
\newcolumntype{L}[1]{>{\raggedright\let\newline\\\arraybackslash\hspace{0pt}}m{#1}}
\newcolumntype{C}[1]{>{\centering\let\newline\\\arraybackslash\hspace{0pt}}m{#1}}
\newcolumntype{R}[1]{>{\raggedleft\let\newline\\\arraybackslash\hspace{0pt}}m{#1}}

\lstdefinelanguage{json}{
    basicstyle=\small\ttfamily,
    numbers=right,
    numberstyle=\scriptsize,
    stepnumber=1,
    numbersep=4pt,
    tabsize=4,
    showstringspaces=false,
    breaklines=true,
    frame=lines,
    belowskip=2em,
    aboveskip=2em,
    keywordstyle=\color{blue},
    commentstyle=\color{dkgreen},
    backgroundcolor=\color{background},
    literate=
     *{0}{{{\color{numb}0}}}{1}
      {1}{{{\color{numb}1}}}{1}
      {2}{{{\color{numb}2}}}{1}
      {3}{{{\color{numb}3}}}{1}
      {4}{{{\color{numb}4}}}{1}
      {5}{{{\color{numb}5}}}{1}
      {6}{{{\color{numb}6}}}{1}
      {7}{{{\color{numb}7}}}{1}
      {8}{{{\color{numb}8}}}{1}
      {9}{{{\color{numb}9}}}{1}
      {:}{{{\color{punct}{:}}}}{1}
      {,}{{{\color{punct}{,}}}}{1}
      {\{}{{{\color{delim}{\{}}}}{1}
      {\}}{{{\color{delim}{\}}}}}{1}
      {[}{{{\color{delim}{[}}}}{1}
      {]}{{{\color{delim}{]}}}}{1},
}

\lstdefinelanguage{Python}{
 keywords={typeof, from, import, null, catch, switch, in, int, str, float, self},
     basicstyle=\footnotesize\ttfamily,
    numbers=right,
    numberstyle=\scriptsize,
    stepnumber=1,
    numbersep=4pt,
    tabsize=4,
    showstringspaces=false,
    breaklines=true,
    frame=lines,
    belowskip=2em,
    aboveskip=2em,
    keywordstyle=\color{blue},
    %commentstyle=\color{dkgreen},
    backgroundcolor=\color{background},
% keywordstyle=\color{ForestGreen}\bfseries,
 ndkeywords={boolean, throw, import},
 ndkeywords={return, class, if ,elif, endif, while, do, else, True, False , catch, def},
 ndkeywordstyle=\color{BrickRed}\bfseries,
 identifierstyle=\color{black},
 sensitive=false,
comment=[l]{\#},
 morecomment=[s]{/*}{*/},
 commentstyle=\color{purple}\ttfamily,
 stringstyle=\color{red}\ttfamily,
}

%
\usepackage{nameref}
%\includeonly{./frontmatter/title, ./chapters/chapter_intro/intro}
%\includeonly{./frontmatter/title, ./frontmatter/abstract, ./chapters/chapter_intro/intro, ./chapters/chapter_conclusiones/conclusiones}

%EPS TO PDF
\usepackage{epstopdf}
\epstopdfsetup{update} % only regenerate pdf files when eps file is newer

\begin{document}
% UoB guidlines:
%
% Preliminary pages
%
% The five preliminary pages must be the Title Page, Abstract, Dedication
% and Acknowledgements, Author's Declaration and Table of Contents.
% These should be single-sided.
%
% Table of contents, list of tables and illustrative material
%
% The table of contents must list, with page numbers, all chapters,
 % sections and subsections, the list of references, bibliography, list of
% abbreviations and appendices. The list of tables and illustrations
% should follow the table of contents, listing with page numbers the
% tables, photographs, diagrams, etc., in the order in which they appear
% in the text.
%
\frontmatter
\pagenumbering{roman}
%
%
% UoB guidelines:
% 
% At the top of the title page, within the margins, the dissertation should give the title and, if 
% necessary, sub-title and volume number. If the dissertation is in a language other than English, the 
% title must be given in that language and in English. The full name of the author should be in the 
% centre of the page. At the bottom centre should be the words ?A dissertation submitted to the 
% University of Bristol in accordance with the requirements for award of the degree of ? in the 
% Faculty of ...?, with the name of the school and month and year of submission. The word count of 
% the dissertation (text only) should be entered at the bottom right-hand side of the page.
%
%
\setlrmarginsandblock{3.5cm}{2.5cm}{*}
\setulmarginsandblock{3cm}{*}{1}
\checkandfixthelayout 
\begin{titlingpage}
\begin{SingleSpace}
\calccentering{\unitlength} 
\begin{adjustwidth*}{\unitlength}{-\unitlength}
%\vspace*{13mm}
\begin{center}
\rule[0.5ex]{\linewidth}{2pt}\vspace*{-\baselineskip}\vspace*{3.2pt}
\rule[0.5ex]{\linewidth}{1pt}\\[\baselineskip]
{\HUGE Titulo de ejemplo de una tesis,\newline doctoral muy importante\newline y relevante para alguien}\\[4mm]
%{\Large \textit{}}\\
\rule[0.5ex]{\linewidth}{1pt}\vspace*{-\baselineskip}\vspace{3.2pt}
\rule[0.5ex]{\linewidth}{2pt}\\
\vspace{5mm}
{\large Por}\\
\vspace{5mm}
{\large\textsc{Menganito de Tal y Pascual}}\\
\vspace{5mm}
\includegraphics[height=7cm]{./logos/uam.png}\\
%\includegraphics[width=6cm]{logos/eps.jpg}\\
{
\vspace{2em}
\textsc{\HUGE Universidad Autónoma de Madrid}\\
\vspace{1.5em}
\textsc{\Large Escuela Politécnica Superior}\\
\vspace{1em}
\large Departamento de ...\\
\vspace{1em}
TESIS DOCTORAL\\
\vspace{2.5em}
\large\textsc{Mes de un determinado año}
}
\vspace{3.5em}
\end{center}
\begin{flushright}
{Director: Sr. Importante}
\end{flushright}
\end{adjustwidth*}
\end{SingleSpace}
\end{titlingpage}
%\restoregeometry

%\setlrmarginsandblock{3cm}{*}{0.7}
%\setulmarginsandblock{*}{3cm}{1}
%\checkandfixthelayout 
\restoregeometry

\clearemptydoublepage
%
\include{./frontmatter/abstract}
\clearemptydoublepage
%

\chapter*{Dedication and acknowledgements}
\begin{SingleSpace}
\initial{H}ere goes the dedication.
\end{SingleSpace}
\clearpage
\clearemptydoublepage
%
\include{./frontmatter/declaration}
\clearemptydoublepage

\chapter*{Frase}
\begin{SingleSpace}
\initial{S}er o no ser…
\begin{center}
\textit{[...]}
\end{center}

Lorem ipsum dolor sit amet, consectetur adipiscing elit. Phasellus cursus sed arcu sit amet pulvinar. Sed efficitur sem risus. Nam sit amet nisi nisi. In accumsan leo eu lorem condimentum interdum. Praesent blandit velit ipsum, vitae sollicitudin ex fringilla quis. Aliquam venenatis vel nibh a accumsan. Vivamus nulla diam, sollicitudin a leo in, lobortis tincidunt nisi. Fusce tellus ex, semper vel tempus non, viverra a dui. Praesent vel leo suscipit, feugiat lacus eget, posuere nulla. Maecenas sodales orci mi, eu pulvinar risus vehicula eget. Vivamus aliquet turpis vitae nibh tristique, in maximus orci tempus. Pellentesque at est dictum ante tristique feugiat a nec nulla. In fringilla dignissim dui, vel accumsan sem semper quis.
\begin{center}
\textit{[...]}
\end{center}

Duis non cursus eros. Nam pretium molestie nibh, vitae venenatis nulla tempus ut. Vestibulum ante ipsum primis in faucibus orci luctus et ultrices posuere cubilia Curae; Duis varius, odio in sollicitudin dignissim, sapien lectus bibendum enim, at ultrices lacus ipsum vitae sapien. Vivamus venenatis, augue ut tempor porta, nunc ex sodales felis, et tristique turpis enim at sapien. Ut dignissim dictum orci, sed gravida elit accumsan sed. Quisque porttitor enim urna, quis suscipit ex feugiat ac. Donec posuere vehicula tellus, vel ullamcorper justo rutrum id. Quisque vel euismod tellus. Suspendisse ut iaculis arcu. Suspendisse at magna non arcu placerat malesuada ut sit amet urna.

\begin{center}
\textit{[...]}
\end{center}

Curabitur varius laoreet convallis. Morbi pharetra et diam ut ullamcorper. Duis sollicitudin tortor metus, nec convallis lorem dictum vitae. Suspendisse mollis non lacus vel efficitur. Sed sed neque pulvinar, ullamcorper tellus at, aliquam elit. Pellentesque efficitur turpis sed ex convallis, eget tempor enim cursus. Morbi et bibendum erat.

 \looseness=-1

\begin{flushright}
\textit{Menganito}
\end{flushright}
\end{SingleSpace}
\clearpage
\clearemptydoublepage
%
\renewcommand{\contentsname}{Tabla de Contenidos}
\maxtocdepth{subsection}
\tableofcontents*
\addtocontents{toc}{\par\nobreak \mbox{}\hfill{\bf Página}\par\nobreak}
\clearemptydoublepage
%
\listoftables
\addtocontents{lot}{\par\nobreak\textbf{{\scshape Tabla} \hfill Página}\par\nobreak}
\clearemptydoublepage
%
\listoffigures
\addtocontents{lof}{\par\nobreak\textbf{{\scshape Figura} \hfill Página}\par\nobreak}
\clearemptydoublepage
%
\lstlistoflistings
\addtocontents{lol}{\par\nobreak \mbox{}\hfill{\bf Página}\par\nobreak}
\clearemptydoublepage

%
%
% The bulk of the document is delegated to these chapter files in
% subdirectories.

\mainmatter

%INTRO
\chapter{Introducción}
\label{chap:intro}

\initial{L}a era de... Lorem ipsum dolor sit amet, consectetur adipiscing elit. Phasellus cursus sed arcu sit amet pulvinar. Sed efficitur sem risus. Nam sit amet nisi nisi. In accumsan leo eu lorem condimentum interdum. Praesent blandit velit ipsum, vitae sollicitudin ex fringilla quis. Aliquam venenatis vel nibh a accumsan. Vivamus nulla diam, sollicitudin a leo in, lobortis tincidunt nisi. Fusce tellus ex, semper vel tempus non, viverra a dui. Praesent vel leo suscipit, feugiat lacus eget, posuere nulla. Maecenas sodales orci mi, eu pulvinar risus vehicula eget. Vivamus aliquet turpis vitae nibh tristique, in maximus orci tempus. Pellentesque at est dictum ante tristique feugiat a nec nulla. In fringilla dignissim dui, vel accumsan sem semper quis.

Lorem ipsum dolor sit amet, consectetur adipiscing elit. Phasellus cursus sed arcu sit amet pulvinar. Sed efficitur sem risus. Nam sit amet nisi nisi. In accumsan leo eu lorem condimentum interdum. Praesent blandit velit ipsum, vitae sollicitudin ex fringilla quis. Aliquam venenatis vel nibh a accumsan. Vivamus nulla diam, sollicitudin a leo in, lobortis tincidunt nisi. Fusce tellus ex, semper vel tempus non, viverra a dui. Praesent vel leo suscipit, feugiat lacus eget, posuere nulla. Maecenas sodales orci mi, eu pulvinar risus vehicula eget. Vivamus aliquet turpis vitae nibh tristique, in maximus orci tempus. Pellentesque at est dictum ante tristique feugiat a nec nulla. In fringilla dignissim dui, vel accumsan sem semper quis.

\section{Motivación}

Lorem ipsum dolor sit amet, consectetur adipiscing elit. Phasellus cursus sed arcu sit amet pulvinar. Sed efficitur sem risus. Nam sit amet nisi nisi. In accumsan leo eu lorem condimentum interdum. Praesent blandit velit ipsum, vitae sollicitudin ex fringilla quis. Aliquam venenatis vel nibh a accumsan. Vivamus nulla diam, sollicitudin a leo in, lobortis tincidunt nisi. Fusce tellus ex, semper vel tempus non, viverra a dui. Praesent vel leo suscipit, feugiat lacus eget, posuere nulla. Maecenas sodales orci mi, eu pulvinar risus vehicula eget. Vivamus aliquet turpis vitae nibh tristique, in maximus orci tempus. Pellentesque at est dictum ante tristique feugiat a nec nulla. In fringilla dignissim dui, vel accumsan sem semper quis.

Lorem ipsum dolor sit amet, consectetur adipiscing elit. Phasellus cursus sed arcu sit amet pulvinar. Sed efficitur sem risus. Nam sit amet nisi nisi. In accumsan leo eu lorem condimentum interdum. Praesent blandit velit ipsum, vitae sollicitudin ex fringilla quis. Aliquam venenatis vel nibh a accumsan. Vivamus nulla diam, sollicitudin a leo in, lobortis tincidunt nisi. Fusce tellus ex, semper vel tempus non, viverra a dui. Praesent vel leo suscipit, feugiat lacus eget, posuere nulla. Maecenas sodales orci mi, eu pulvinar risus vehicula eget. Vivamus aliquet turpis vitae nibh tristique, in maximus orci tempus. Pellentesque at est dictum ante tristique feugiat a nec nulla. In fringilla dignissim dui, vel accumsan sem semper quis.

Lorem ipsum dolor sit amet, consectetur adipiscing elit. Phasellus cursus sed arcu sit amet pulvinar. Sed efficitur sem risus. Nam sit amet nisi nisi. In accumsan leo eu lorem condimentum interdum. Praesent blandit velit ipsum, vitae sollicitudin ex fringilla quis. Aliquam venenatis vel nibh a accumsan. Vivamus nulla diam, sollicitudin a leo in, lobortis tincidunt nisi. Fusce tellus ex, semper vel tempus non, viverra a dui. Praesent vel leo suscipit, feugiat lacus eget, posuere nulla. Maecenas sodales orci mi, eu pulvinar risus vehicula eget. Vivamus aliquet turpis vitae nibh tristique, in maximus orci tempus. Pellentesque at est dictum ante tristique feugiat a nec nulla. In fringilla dignissim dui, vel accumsan sem semper quis.

%%%CONCLUSIONES
\include{./chapters/chapter_conclusiones/conclusiones}
\clearemptydoublepage
%
%
% And the appendix goes here
\appendix

\chapter{Appendix A}
\label{app:app01}

\initial{B}egins an appendix

\begin{figure}[p]
\centering
\includegraphics[width=\textwidth]{logos/eps.jpg}
\label{dia:eps}
\end{figure}








\
\clearemptydoublepage
%
% Apparently the guidelines don't say anything about citations or
% bibliography styles so I guess we can use anything.
\backmatter
\bibliographystyle{siam}
\refstepcounter{chapter}
\begin{thebibliography}{Bibliografía}
{\footnotesize
\setlength{\bibhang}{0em}

%INTRO

\bibitem{ref:unaref}
Fulano, Mengano. \emph{Algun titulo de algun articulo} (año) Revista\\
\url{https://example.com}

% etc
}%end footnotesize
\end{thebibliography}
\clearemptydoublepage
%
% Add index
%\printindex
%
\end{document}